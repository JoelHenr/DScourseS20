\documentclass{article}
\usepackage[utf8]{inputenc}

\title{PS12}
\author{Joel Henriques}
\date{April 2020}

\usepackage{natbib}
\usepackage{graphicx}

\begin{document}

\maketitle

\section{Questions}
\begin{itemize}
    \item  The results on the first graph still makes sense except for the log wages unless the units are in dollars per hour. The idea of have a child on the 75 and 100 percentile is equal to 1. Therefore, this should check out. So, at this point it is nothing too much to worry about. 
    \item I think this is most likley MNAR, because there seems to be a range missing in the data for log wages between 1-1.5. 
    \item The listwise delection method was the closest to the true value. The heckman selection was the furthest off whereas the imputation method was still off but did better. Therefore, I guess deleting your data works the best!(Just kidding)
    \item The last chart definitely seems realistic. It reports that having children takes away from the wages, which is true because you have to provide for another person which should decrease wages. Also, being married could increase wages because your significant other makes an income which should increase total revenue per person. Thus, the data seems realistic. 
\end{itemize}
\begin{figure}[h!]
\centering
\includegraphics[scale=.50]{1.png}
\includegraphics[scale=.50]{2.png}
\includegraphics[scale=.55]{3.png}
\caption{The Tables}
\label{fig:universe}
\end{figure}

\end{document}
