\documentclass{article}
\usepackage[utf8]{inputenc}

\title{PS6}
\author{Joel Henriques }
\date{March 2020}

\usepackage{natbib}
\usepackage{graphicx}

\begin{document}

\maketitle

\section{Clean Data and Meaning}
I downloaded the csv file from Fred by used code to read the file from my desktop into R. Then, I checked if there were any "Na" within the data set. After, I wanted to make sure that the data set was formatted correctly. In addition to the GDP and Date variables, I created two more columns in order to do a percentage change calculation. The first column created was the differences between the current GDP and a lagged GDP. The second column used the first column differences and divides it by the lagged GDP and multiplied by 100 in order to calculate the percentage change. 


These charts show the percentage changes of GDP in the United States over time. It does an excellent job of showing the impacts of the 2008-2009 recession on GDP growth. It is also shocking that the United States still has not recovered to the peak of its GDP growth. 

\begin{figure}[hbt!]
\centering
\includegraphics[scale=.33]{Line Clean.png}
\caption{Change in GDP over Time}
\label{fig1:Line Clean}


\includegraphics[scale=.33]{Clean Stata.png}
\caption{Change in GDP over Time}
\label{fig2:Clean Stata}


\includegraphics[scale=.33]{Line Economist.png}
\caption{Change in GDP over Time}
\label{fig2:Line Economist}
\end{figure}


\end{document}
